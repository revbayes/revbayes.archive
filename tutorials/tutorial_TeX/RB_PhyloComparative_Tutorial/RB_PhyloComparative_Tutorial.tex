\documentclass[usletter]{article}
\usepackage[latin1]{inputenc}
%\usepackage[french]{babel}
%\usepackage{t1enc}
%\usepackage[francais]{babel}

\usepackage{vmargin}
\usepackage{amssymb,amstext,amsmath}
\usepackage{hyperref}
\usepackage{epsfig}
\usepackage{array}
\usepackage{color}
\usepackage{xspace}

\usepackage{latexsym}

\usepackage{tikz} 

\usepgflibrary{shapes.misc} 

\newcommand{\cmd}[1]{\texttt{#1}}

%\newtheorem{command}{Command}{\bfseries}{\normalfont}

\usepackage{natbib}


\renewcommand{\baselinestretch}{1.5}

\title{RevBayes -- Phylogenies and the comparative method}

\author{Nicolas Lartillot}


\begin{document}
\maketitle

\documentclass[usletter]{article}
\usepackage[latin1]{inputenc}
%\usepackage[french]{babel}
%\usepackage{t1enc}
%\usepackage[francais]{babel}

\usepackage{vmargin}
\usepackage{amssymb,amstext,amsmath}
\usepackage{hyperref}
\usepackage{epsfig}
\usepackage{array}
\usepackage{color}
\usepackage{xspace}

\usepackage{latexsym}

\usepackage{tikz} 

\usepgflibrary{shapes.misc} 

\newcommand{\cmd}[1]{\texttt{#1}}

%\newtheorem{command}{Command}{\bfseries}{\normalfont}

\usepackage{natbib}


\renewcommand{\baselinestretch}{1.5}

\title{RevBayes -- Phylogenies and the comparative method}

\author{Nicolas Lartillot}


\begin{document}
\maketitle

\documentclass[usletter]{article}
\usepackage[latin1]{inputenc}
%\usepackage[french]{babel}
%\usepackage{t1enc}
%\usepackage[francais]{babel}

\usepackage{vmargin}
\usepackage{amssymb,amstext,amsmath}
\usepackage{hyperref}
\usepackage{epsfig}
\usepackage{array}
\usepackage{color}
\usepackage{xspace}

\usepackage{latexsym}

\usepackage{tikz} 

\usepgflibrary{shapes.misc} 

\newcommand{\cmd}[1]{\texttt{#1}}

%\newtheorem{command}{Command}{\bfseries}{\normalfont}

\usepackage{natbib}


\renewcommand{\baselinestretch}{1.5}

\title{RevBayes -- Phylogenies and the comparative method}

\author{Nicolas Lartillot}


\begin{document}
\maketitle

\documentclass[usletter]{article}
\usepackage[latin1]{inputenc}
%\usepackage[french]{babel}
%\usepackage{t1enc}
%\usepackage[francais]{babel}

\usepackage{vmargin}
\usepackage{amssymb,amstext,amsmath}
\usepackage{hyperref}
\usepackage{epsfig}
\usepackage{array}
\usepackage{color}
\usepackage{xspace}

\usepackage{latexsym}

\usepackage{tikz} 

\usepgflibrary{shapes.misc} 

\newcommand{\cmd}[1]{\texttt{#1}}

%\newtheorem{command}{Command}{\bfseries}{\normalfont}

\usepackage{natbib}


\renewcommand{\baselinestretch}{1.5}

\title{RevBayes -- Phylogenies and the comparative method}

\author{Nicolas Lartillot}


\begin{document}
\maketitle

\input{RB_PhyloComparative_Tutorial}

\bibliographystyle{natbib}

\bibliography{allbib}

\end{document}  

\subsection*{Testing diversification models}

make an integrated diversification studies and comparative analysis: combining phylogenetic estimation, dating, reconstruction of body size evolution and test of a diversification model in the case of placentals: in their case, a good model to test would be a skyline birth death or, perhaps more interestingly, a burst and stasis model, with the burst occurring right at KT.

\subsection*{Comparative analysis and incomplete lineage sorting}

Let the generation time $t$, the mutation rate per generation $u$ and the effective population size $N_e$ be a joint multivariate (log-normal) Brownian process along the species tree. Then, plug $(N_e t)^{-1}$ as the coalescence rate within the ILS model, and $u / t$ as the substitution rate per calendar unit of time in the substitution model running along gene genealogies. Of course, all this can be correlated with body size and life-history traits. Predictions are: $N_e$ correlates negatively, $t$ positively and $u$ positively (but $u / t$ negatively) with body-size. Nice allometric analysis of ILS...



\bibliographystyle{natbib}

\bibliography{allbib}

\end{document}  

\subsection*{Testing diversification models}

make an integrated diversification studies and comparative analysis: combining phylogenetic estimation, dating, reconstruction of body size evolution and test of a diversification model in the case of placentals: in their case, a good model to test would be a skyline birth death or, perhaps more interestingly, a burst and stasis model, with the burst occurring right at KT.

\subsection*{Comparative analysis and incomplete lineage sorting}

Let the generation time $t$, the mutation rate per generation $u$ and the effective population size $N_e$ be a joint multivariate (log-normal) Brownian process along the species tree. Then, plug $(N_e t)^{-1}$ as the coalescence rate within the ILS model, and $u / t$ as the substitution rate per calendar unit of time in the substitution model running along gene genealogies. Of course, all this can be correlated with body size and life-history traits. Predictions are: $N_e$ correlates negatively, $t$ positively and $u$ positively (but $u / t$ negatively) with body-size. Nice allometric analysis of ILS...



\bibliographystyle{natbib}

\bibliography{allbib}

\end{document}  

\subsection*{Testing diversification models}

make an integrated diversification studies and comparative analysis: combining phylogenetic estimation, dating, reconstruction of body size evolution and test of a diversification model in the case of placentals: in their case, a good model to test would be a skyline birth death or, perhaps more interestingly, a burst and stasis model, with the burst occurring right at KT.

\subsection*{Comparative analysis and incomplete lineage sorting}

Let the generation time $t$, the mutation rate per generation $u$ and the effective population size $N_e$ be a joint multivariate (log-normal) Brownian process along the species tree. Then, plug $(N_e t)^{-1}$ as the coalescence rate within the ILS model, and $u / t$ as the substitution rate per calendar unit of time in the substitution model running along gene genealogies. Of course, all this can be correlated with body size and life-history traits. Predictions are: $N_e$ correlates negatively, $t$ positively and $u$ positively (but $u / t$ negatively) with body-size. Nice allometric analysis of ILS...



\bibliographystyle{natbib}

\bibliography{allbib}

\end{document}  

\subsection*{Testing diversification models}

make an integrated diversification studies and comparative analysis: combining phylogenetic estimation, dating, reconstruction of body size evolution and test of a diversification model in the case of placentals: in their case, a good model to test would be a skyline birth death or, perhaps more interestingly, a burst and stasis model, with the burst occurring right at KT.

\subsection*{Comparative analysis and incomplete lineage sorting}

Let the generation time $t$, the mutation rate per generation $u$ and the effective population size $N_e$ be a joint multivariate (log-normal) Brownian process along the species tree. Then, plug $(N_e t)^{-1}$ as the coalescence rate within the ILS model, and $u / t$ as the substitution rate per calendar unit of time in the substitution model running along gene genealogies. Of course, all this can be correlated with body size and life-history traits. Predictions are: $N_e$ correlates negatively, $t$ positively and $u$ positively (but $u / t$ negatively) with body-size. Nice allometric analysis of ILS...

